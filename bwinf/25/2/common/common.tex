\documentclass [12pt] {article}
\usepackage{ngerman} %Deutsche Sprache (Umlaute...)
\usepackage[latin1]{inputenc} %Umlaute im Text erkennen

\usepackage{amsfonts}
\usepackage{amsmath}
\usepackage[dvips]{graphicx} 
\DeclareGraphicsExtensions{.eps, .png, .jpg}
%\DeclareGraphicsRule{.png}{eps}{}{convert #1 eps:-}
%\usepackage{pgf}
\usepackage{fancyhdr}
\setlength{\headheight}{28pt}

\usepackage{lgrind} %nice code
\usepackage{bb_code} %nice code
\usepackage{verbatim}

%\usepackage[paper=a4paper,left=35mm,right=35mm,top=35mm,bottom=35mm]{geometry} 

\newcommand{\n}{\newline}
\renewcommand{\b}{\textbf}
\renewcommand{\u}{\underline}
%\newcommand{	}{\par\noindent\n}
\newcommand{\mN}{\mathbb{N}}


%%Strange characters
\newcommand{\scSHARP}{\char'043} 	

\title {Allgemeines}
\pagestyle{fancyplain}
\lhead{Benito van der Zander, 25.834.01}
\rhead{}
\fancyfoot{}

\begin{document}

\section*{}
\section*{Allgemeines}
\section*{}
\setlength{\parindent}{0pt} 
\setlength{\parskip}{17pt} 

Ich habe die Aufgabe 2 "`Bidoku"' und die Aufgabe 3 "`Folgen"' bearbeitet.

Alle Programme sind in FreePascal geschrieben. Die compilierten Versionen auf der CD laufen unter Windows, es sollte aber auch m�glich sein, den Quellcode unter Linux zu compilern und die Programme dort zu benutzen.

Die Einsendung enth�lt:

\begin{enumerate}
\item 1 Seite Allgemeines (diese hier)
\item 1 Briefumschlag (au�en)
\item 62 Seiten zur Aufgabe 2 "`Bidoku"'
\item 56 Seiten zur Aufgabe 3 "`Folgen"'
\item 1 Seite zum MCI-Sonderpreis
\item 1 CD
\end{enumerate}



\newpage
\section*{}
\section*{MCI-Sonderpreis}
\section*{}

Hiermit m�chte ich mein Programm zum Bidoku f�r den MCI-Sonderpreis anmelden.

Bei dem Design habe ich mich auf das Wesentliche konzentriert und so besteht das Fenster im Wesentlichen aus einem Bidokufeld.

Die Funktionen lassen sich alle �ber eine klassische Windowsmen�leiste erreichen und sind thematisch in Funktionen, die zur Verwaltung mehrerer Bidokus (also Erzeugung, Speicherung, und �hnliches) dienen, und in Funktionen, die das aktuelle Bidoku ver�ndern (z.B.: l�sen) oder analysieren (z.B.:entschl�sseln) unterteilt.

Alle h�ufig ben�tigten Befehle lassen sich mittels Tastenkombinationen schnell erreichen, wodurch das Programm effizient zu benutzen ist.\\
Die seltener benutzten Befehle sind trotzdem noch einigerma�en schnell, �ber die Men�s zu erreichen, da diese nicht verschachtelt sind.

Die Verwendung mehrere Bidokus wird dadurch �bersichtlicher, dass die Liste der vorhandenen Bidokus automatisch ein- und ausgeblendet wird.


Durch die M�glichkeit alle Schritte, die zur Erzeugung eines Bidokus gef�hrt haben, anzuzeigen, kann man auch gut den internen Ablauf des Programmes erkennen.

\end{document}
